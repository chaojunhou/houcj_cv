\documentclass[11pt,a4paper,nolmodern]{moderncv}
\usepackage[noindent,UTF8]{ctex}
\usepackage{info}

\title{北京大学}


\begin{document}
\setmainfont{Minion Pro}
\setsansfont{Myriad Pro}

\hyphenpenalty=10000
\maketitle

\section{教育背景}
\tlcventry{2013}{2016}{北京大学,计算机应用技术专业,理学硕士,互联网方向}{}{}{}{
\begin{itemize}
  \item 北京大学 2014-2015 学年二等学业奖学金
  \item GPA: 3.69/4.0
\end{itemize}}

\tlcventry{2009}{2013}{天津理工大学,计算机科学与技术专业,工学硕士}{}{}{}{
\begin{itemize}
  \item 2010年天津市数学竞赛一等奖
  \item 2012年全国大学生数学建模竞赛天津市二等奖
\end{itemize}}



\section{科研项目(\href{https://github.com/chaojunhou}{GitHub链接})}
\tlcventry{2014}{0}{基于ndnSIM的网络拥塞控制算法}{}{}{}{
\begin{itemize}
 \item ndnSIM是基于ns-3的网络模拟器
 \item 通过在网络层加入队列管理(主动队列管理),对兴趣包进行调度来减少网络拥塞
 \item 开发平台:Linux, 编程语言: C++
\end{itemize}}
\vspace{0.5em}
\tlcventry{2013}{0}{知乎收藏问题答案推送到Kindle阅读器}{}{}{}{
\begin{itemize}
 \item 使用requests模拟浏览器登录知乎网站
 \item 将爬取到的知乎收藏问答页面的答案以邮件的形式推送到kindle
 \item 开发平台:Linux, 编程语言: Python
\end{itemize}}
\vspace{0.5em}
\tlcventry{2013}{0}{基于Django和BAE的微信禾乐服务号的开发}{}{}{}{
\begin{itemize}
 \item 参与开发了微信API和JSON数据格式
 \item 参与开发了MVC(MTV)框架模式的使用
 \item 开发平台:Linux, 编程语言: Python
\end{itemize}}
\vspace{0.5em}
\tlcventry{2012}{0}{Windows下的流量监控系统}{}{}{}{
\begin{itemize}
 \item 通过系统接口获取本机的网卡和网络信息
 \item 以图形化的形式显示获取的网络数据
 \item 开发平台:Windows, 编程语言: C\#
\end{itemize}}

\section{专业技能}
\cvcomputer{熟悉}{Python、C/C++、Linux,了解Java、C\# 、Scheme}{}{}
\cvcomputer{熟悉}{TCP/IP协议,网络拥塞控制算法}{}{}
\cvcomputer{了解}{Socket编程、Shell脚本、Git、有多线程编程经验}{}{}
\cvcomputer{英语}{通过CET-4和CET-6,有较强英文文献和文档阅读能力}{}{}

         

\end{document}

